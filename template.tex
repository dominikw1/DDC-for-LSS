%% bare_conf.tex
%% V1.4
%% 2012/12/27
%% by Michael Shell
%% See:
%% http://www.michaelshell.org/
%% for current contact information.
%%
%% This is a skeleton file demonstrating the use of IEEEtran.cls
%% (requires IEEEtran.cls version 1.8 or later) with an IEEE conference paper.
%%
%% Support sites:
%% http://www.michaelshell.org/tex/ieeetran/
%% http://www.ctan.org/tex-archive/macros/latex/contrib/IEEEtran/
%% and
%% http://www.ieee.org/
%%*************************************************************************
%% Legal Notice:
%% This code is offered as-is without any warranty either expressed or
%% implied; without even the implied warranty of MERCHANTABILITY or
%% FITNESS FOR A PARTICULAR PURPOSE! 
%% User assumes all risk.
%% In no event shall IEEE or any contributor to this code be liable for
%% any damages or losses, including, but not limited to, incidental,
%% consequential, or any other damages, resulting from the use or misuse
%% of any information contained here.
%%
%% All comments are the opinions of their respective authors and are not
%% necessarily endorsed by the IEEE.
%%
%% This work is distributed under the LaTeX Project Public License (LPPL)
%% ( http://www.latex-project.org/ ) version 1.3, and may be freely used,
%% distributed and modified. A copy of the LPPL, version 1.3, is included
%% in the base LaTeX documentation of all distributions of LaTeX released
%% 2003/12/01 or later.
%% Retain all contribution notices and credits.
%% ** Modified files should be clearly indicated as such, including  **
%% ** renaming them and changing author support contact information. **
%%
%% File list of work: IEEEtran.cls, IEEEtran_HOWTO.pdf, bare_adv.tex,
%%                    bare_conf.tex, bare_jrnl.tex, bare_jrnl_compsoc.tex,
%%                    bare_jrnl_transmag.tex
%%*************************************************************************

% *** Authors should verify (and, if needed, correct) their LaTeX system  ***
% *** with the testflow diagnostic prior to trusting their LaTeX platform ***
% *** with production work. IEEE's font choices can trigger bugs that do  ***
% *** not appear when using other class files.                            ***
% The testflow support page is at:
% http://www.michaelshell.org/tex/testflow/



% Note that the a4paper option is mainly intended so that authors in
% countries using A4 can easily print to A4 and see how their papers will
% look in print - the typesetting of the document will not typically be
% affected with changes in paper size (but the bottom and side margins will).
% Use the testflow package mentioned above to verify correct handling of
% both paper sizes by the user's LaTeX system.
%
% Also note that the "draftcls" or "draftclsnofoot", not "draft", option
% should be used if it is desired that the figures are to be displayed in
% draft mode.
%
\documentclass[conference]{IEEEtran}
% Add the compsoc option for Computer Society conferences.
%
% If IEEEtran.cls has not been installed into the LaTeX system files,
% manually specify the path to it like:
% \documentclass[conference]{../sty/IEEEtran}



% Some very useful LaTeX packages include:
% (uncomment the ones you want to load)


% For directly writing german umlauts uncomment the appropriate line for
% your operating system:
% Windows:
% \usepackage[ansinew]{inputenc}
% Linux:
\usepackage[latin1]{inputenc}
% Mac
% \usepackage[applemac]{inputenc}
% If none of the above lines work you can also try the following:
% \usepackage[utf8]{inputenc}



% *** MISC UTILITY PACKAGES ***
%
%\usepackage{ifpdf}
% Heiko Oberdiek's ifpdf.sty is very useful if you need conditional
% compilation based on whether the output is pdf or dvi.
% usage:
% \ifpdf
%   % pdf code
% \else
%   % dvi code
% \fi
% The latest version of ifpdf.sty can be obtained from:
% http://www.ctan.org/tex-archive/macros/latex/contrib/oberdiek/
% Also, note that IEEEtran.cls V1.7 and later provides a builtin
% \ifCLASSINFOpdf conditional that works the same way.
% When switching from latex to pdflatex and vice-versa, the compiler may
% have to be run twice to clear warning/error messages.






% *** CITATION PACKAGES ***
%
%\usepackage{cite}
% cite.sty was written by Donald Arseneau
% V1.6 and later of IEEEtran pre-defines the format of the cite.sty package
% \cite{} output to follow that of IEEE. Loading the cite package will
% result in citation numbers being automatically sorted and properly
% "compressed/ranged". e.g., [1], [9], [2], [7], [5], [6] without using
% cite.sty will become [1], [2], [5]--[7], [9] using cite.sty. cite.sty's
% \cite will automatically add leading space, if needed. Use cite.sty's
% noadjust option (cite.sty V3.8 and later) if you want to turn this off
% such as if a citation ever needs to be enclosed in parenthesis.
% cite.sty is already installed on most LaTeX systems. Be sure and use
% version 4.0 (2003-05-27) and later if using hyperref.sty. cite.sty does
% not currently provide for hyperlinked citations.
% The latest version can be obtained at:
% http://www.ctan.org/tex-archive/macros/latex/contrib/cite/
% The documentation is contained in the cite.sty file itself.


\usepackage{amsmath}
\usepackage{amssymb}



% *** GRAPHICS RELATED PACKAGES ***
%
\ifCLASSINFOpdf
  \usepackage[pdftex]{graphicx}
  % declare the path(s) where your graphic files are
  % \graphicspath{{../pdf/}{../jpeg/}}
  % and their extensions so you won't have to specify these with
  % every instance of \includegraphics
  % \DeclareGraphicsExtensions{.pdf,.jpeg,.png}
\else
  % or other class option (dvipsone, dvipdf, if not using dvips). graphicx
  % will default to the driver specified in the system graphics.cfg if no
  % driver is specified.
  \usepackage[dvips]{graphicx}
  % declare the path(s) where your graphic files are
  % \graphicspath{{../eps/}}
  % and their extensions so you won't have to specify these with
  % every instance of \includegraphics
  % \DeclareGraphicsExtensions{.eps}
\fi
% graphicx was written by David Carlisle and Sebastian Rahtz. It is
% required if you want graphics, photos, etc. graphicx.sty is already
% installed on most LaTeX systems. The latest version and documentation
% can be obtained at: 
% http://www.ctan.org/tex-archive/macros/latex/required/graphics/
% Another good source of documentation is "Using Imported Graphics in
% LaTeX2e" by Keith Reckdahl which can be found at:
% http://www.ctan.org/tex-archive/info/epslatex/
%
% latex, and pdflatex in dvi mode, support graphics in encapsulated
% postscript (.eps) format. pdflatex in pdf mode supports graphics
% in .pdf, .jpeg, .png and .mps (metapost) formats. Users should ensure
% that all non-photo figures use a vector format (.eps, .pdf, .mps) and
% not a bitmapped formats (.jpeg, .png). IEEE frowns on bitmapped formats
% which can result in "jaggedy"/blurry rendering of lines and letters as
% well as large increases in file sizes.
%
% You can find documentation about the pdfTeX application at:
% http://www.tug.org/applications/pdftex





% *** MATH PACKAGES ***
%
%\usepackage[cmex10]{amsmath}
% A popular package from the American Mathematical Society that provides
% many useful and powerful commands for dealing with mathematics. If using
% it, be sure to load this package with the cmex10 option to ensure that
% only type 1 fonts will utilized at all point sizes. Without this option,
% it is possible that some math symbols, particularly those within
% footnotes, will be rendered in bitmap form which will result in a
% document that can not be IEEE Xplore compliant!
%
% Also, note that the amsmath package sets \interdisplaylinepenalty to 10000
% thus preventing page breaks from occurring within multiline equations. Use:
%\interdisplaylinepenalty=2500
% after loading amsmath to restore such page breaks as IEEEtran.cls normally
% does. amsmath.sty is already installed on most LaTeX systems. The latest
% version and documentation can be obtained at:
% http://www.ctan.org/tex-archive/macros/latex/required/amslatex/math/





% *** SPECIALIZED LIST PACKAGES ***
%
%\usepackage{algorithmic}
% algorithmic.sty was written by Peter Williams and Rogerio Brito.
% This package provides an algorithmic environment fo describing algorithms.
% You can use the algorithmic environment in-text or within a figure
% environment to provide for a floating algorithm. Do NOT use the algorithm
% floating environment provided by algorithm.sty (by the same authors) or
% algorithm2e.sty (by Christophe Fiorio) as IEEE does not use dedicated
% algorithm float types and packages that provide these will not provide
% correct IEEE style captions. The latest version and documentation of
% algorithmic.sty can be obtained at:
% http://www.ctan.org/tex-archive/macros/latex/contrib/algorithms/
% There is also a support site at:
% http://algorithms.berlios.de/index.html
% Also of interest may be the (relatively newer and more customizable)
% algorithmicx.sty package by Szasz Janos:
% http://www.ctan.org/tex-archive/macros/latex/contrib/algorithmicx/


\DeclareMathOperator*{\argmax}{arg\,max}
\DeclareMathOperator*{\argmin}{arg\,min}


% *** ALIGNMENT PACKAGES ***
%
%\usepackage{array}
% Frank Mittelbach's and David Carlisle's array.sty patches and improves
% the standard LaTeX2e array and tabular environments to provide better
% appearance and additional user controls. As the default LaTeX2e table
% generation code is lacking to the point of almost being broken with
% respect to the quality of the end results, all users are strongly
% advised to use an enhanced (at the very least that provided by array.sty)
% set of table tools. array.sty is already installed on most systems. The
% latest version and documentation can be obtained at:
% http://www.ctan.org/tex-archive/macros/latex/required/tools/


% IEEEtran contains the IEEEeqnarray family of commands that can be used to
% generate multiline equations as well as matrices, tables, etc., of high
% quality.




% *** SUBFIGURE PACKAGES ***
%\ifCLASSOPTIONcompsoc
%  \usepackage[caption=false,font=normalsize,labelfont=sf,textfont=sf]{subfig}
%\else
%  \usepackage[caption=false,font=footnotesize]{subfig}
%\fi
% subfig.sty, written by Steven Douglas Cochran, is the modern replacement
% for subfigure.sty, the latter of which is no longer maintained and is
% incompatible with some LaTeX packages including fixltx2e. However,
% subfig.sty requires and automatically loads Axel Sommerfeldt's caption.sty
% which will override IEEEtran.cls' handling of captions and this will result
% in non-IEEE style figure/table captions. To prevent this problem, be sure
% and invoke subfig.sty's "caption=false" package option (available since
% subfig.sty version 1.3, 2005/06/28) as this is will preserve IEEEtran.cls
% handling of captions.
% Note that the Computer Society format requires a larger sans serif font
% than the serif footnote size font used in traditional IEEE formatting
% and thus the need to invoke different subfig.sty package options depending
% on whether compsoc mode has been enabled.
%
% The latest version and documentation of subfig.sty can be obtained at:
% http://www.ctan.org/tex-archive/macros/latex/contrib/subfig/




% *** FLOAT PACKAGES ***
%
%\usepackage{fixltx2e}
% fixltx2e, the successor to the earlier fix2col.sty, was written by
% Frank Mittelbach and David Carlisle. This package corrects a few problems
% in the LaTeX2e kernel, the most notable of which is that in current
% LaTeX2e releases, the ordering of single and double column floats is not
% guaranteed to be preserved. Thus, an unpatched LaTeX2e can allow a
% single column figure to be placed prior to an earlier double column
% figure. The latest version and documentation can be found at:
% http://www.ctan.org/tex-archive/macros/latex/base/


%\usepackage{stfloats}
% stfloats.sty was written by Sigitas Tolusis. This package gives LaTeX2e
% the ability to do double column floats at the bottom of the page as well
% as the top. (e.g., "\begin{figure*}[!b]" is not normally possible in
% LaTeX2e). It also provides a command:
%\fnbelowfloat
% to enable the placement of footnotes below bottom floats (the standard
% LaTeX2e kernel puts them above bottom floats). This is an invasive package
% which rewrites many portions of the LaTeX2e float routines. It may not work
% with other packages that modify the LaTeX2e float routines. The latest
% version and documentation can be obtained at:
% http://www.ctan.org/tex-archive/macros/latex/contrib/sttools/
% Do not use the stfloats baselinefloat ability as IEEE does not allow
% \baselineskip to stretch. Authors submitting work to the IEEE should note
% that IEEE rarely uses double column equations and that authors should try
% to avoid such use. Do not be tempted to use the cuted.sty or midfloat.sty
% packages (also by Sigitas Tolusis) as IEEE does not format its papers in
% such ways.
% Do not attempt to use stfloats with fixltx2e as they are incompatible.
% Instead, use Morten Hogholm'a dblfloatfix which combines the features
% of both fixltx2e and stfloats:
%
% \usepackage{dblfloatfix}
% The latest version can be found at:
% http://www.ctan.org/tex-archive/macros/latex/contrib/dblfloatfix/




% *** PDF, URL AND HYPERLINK PACKAGES ***
%
%\usepackage{url}
% url.sty was written by Donald Arseneau. It provides better support for
% handling and breaking URLs. url.sty is already installed on most LaTeX
% systems. The latest version and documentation can be obtained at:
% http://www.ctan.org/tex-archive/macros/latex/contrib/url/
% Basically, \url{my_url_here}.




% *** Do not adjust lengths that control margins, column widths, etc. ***
% *** Do not use packages that alter fonts (such as pslatex).         ***
% There should be no need to do such things with IEEEtran.cls V1.6 and later.
% (Unless specifically asked to do so by the journal or conference you plan
% to submit to, of course. )


% add custom packages
\usepackage{color}
\definecolor{tumblue}{rgb}{0, 0.4, 0.74}

\usepackage[backend=biber,  style=ieee, url=false,doi=false, isbn=false, citestyle=numeric-comp]{biblatex}
\addbibresource{references.bib}


% correct bad hyphenation here
\hyphenation{op-tical net-works semi-conduc-tor}


\begin{document}

% Add the seminar's cover page
\begin{figure*}[!h]

  \includegraphics{./images/IN.pdf} \hfill \includegraphics{./images/tumlogo.pdf}
 
  \vspace*{1cm}
  {\large \textsf{Fakult{\"a}t f{\"u}r Informatik}}\\
  {\large \textsf{Lehrstuhl f{\"u}r Echtzeitsysteme und Robotik}}\\
   

  \vspace*{5cm}
%
%
% TITEL DER ARBEIT
%
%
  {\color{tumblue} \Huge \bf \textsf{Data-Driven Control for Large-Scale Systems: A Survey}}\\  % HIER EINSETZEN!

  \vspace*{1cm}
%
%
% NAME DES STUDENTEN (auf Titelblatt)
%
% 
  {\Large \bf \textsf{Dominik M. Weber}}\\   % HIER EINSETZEN!
 
  \vspace*{8cm}
  {\Large \textsf{Seminar Course \emph{Cyber Physical Systems} WS 2025/2026}}\\
 
  \vspace*{1cm} 
  \begin{tabular}{ll}
%
%
% NAME DES BETREUERS
%
%
    {\Large \bf \textsf{Advisor:}} &
    {\Large \textsf{Yongkuan Zhang}}\\                  % HIER EINSETZEN!
    \\

    {\Large \bf \textsf{Supervisor:}} &
    {\Large \textsf{Prof.~Dr.-Ing. Matthias Althoff}}\\
    \\

%
%
% ABGABETERMIN
%
%
    {\Large \bf \textsf{Submission:}} &
    {\Large \textsf{01. January 2026}}

  \end{tabular}
  
\end{figure*}


%
% paper title
% can use linebreaks \\ within to get better formatting as desired
% Do not put math or special symbols in the title.
\title{Data Driven Control for Large Scale Systems: \\ A Survey}


% author names and affiliations
% use a multiple column layout for up to three different
% affiliations
\author{\IEEEauthorblockN{Dominik M. Weber}
\IEEEauthorblockA{Technical University of Munich\\
Email: dominik.m.weber@gmail.com}}

% conference papers do not typically use \thanks and this command
% is locked out in conference mode. If really needed, such as for
% the acknowledgment of grants, issue a \IEEEoverridecommandlockouts
% after \documentclass

% use for special paper notices
%\IEEEspecialpapernotice{(Invited Paper)}



% make the title area
\maketitle

% As a general rule, do not put math, special symbols or citations
% in the abstract
\begin{abstract}
The abstract goes here.
\end{abstract}

% no keywords




% For peer review papers, you can put extra information on the cover
% page as needed:
% \ifCLASSOPTIONpeerreview
% \begin{center} \bfseries EDICS Category: 3-BBND \end{center}
% \fi
%
% For peerreview papers, this IEEEtran command inserts a page break and
% creates the second title. It will be ignored for other modes.
\IEEEpeerreviewmaketitle



\section{Introduction}
% no \IEEEPARstart
\section{Data Driven Control}



Traditional control approaches largely followed the strategy of first deriving a model
from first principles and afterwards building a controller upon it. 
A premier example of such an approach is MPC, model predictive control.
Model-based control can become intractible or inaccurate for complex systems, though~\cite{DDC_Overview}.

\subsection{Classification}

A fundamental delimination in data driven control can be made between model-based and model-free approaches. 
In model-based methods, a model is learned from data - a process termed system identification (SI) - and, as in traditional control, a controller built upon it.
Examples of techniques using this strategy are Koopman-theory~\cite{Korda_2018} and Gaussian process~\cite{Kocijan01122005} based approaches.

In model-free control, the dynamics of the system are treated as a black box and the controller is directly built from the data~\cite{DDC_Overview}.



\subsection{Advantages}

Data-driven approaches provide a range of benefits compared to traditional model-based ones. 
System identification, the basis for indirect data-driven methods, has
the key advantage of being generally applicable and to large degrees automated, 
whereas a model-based technique requires expert-knowledge for deriving the model from first principles,
the specific result then being domain-specific. SI further provides the practitioner with 
tools to fine-tune the accuracy required, as complexity tends to rise with more accurate models~\cite{behavioralDataDrivenControl}.


\subsection{Disadvantages}

Despite their benefits, data-driven control methods 
face several limitations. 
A central challenge is the heavy reliance on the 
quality and quantity of available data, insufficient or 
noisy datasets having the potential to significantly 
degrade the performance of the resulting controller~\cite{HOU20133}.


%Further, data-driven methods often come with high computational 
%demands, both in terms of learning and real-time operation, 
%which limits their deployment in safety-critical 
%or resource-constrained systems [6]. 

Another drawback is the lack of interpretability and 
transparency compared to physics-based models. 
First-principles models typically allow direct 
insight into the mechanisms driving system behavior, 
black-box controllers derived purely from data can be 
difficult to analyze or verify~\cite{Brunton_Kutz_2022}. 

Finally, issues of robustness and generalization remain 
open problems, as controllers trained on one operational 
regime may fail to extrapolate reliably to unseen 
conditions~\cite{ramadan2025floodgatescontaindeepclimit}.

%For large and complex systems, automatic system identification hence   

\section{Large Scale Systems}


Large Scale Systems do not have a single clear definition.
The term is rather used whenever a system grows unweildly to control with traditional strategies.
It is further often used synonymously with complex system, a system consisting of multiple interacting units 
that exhibits emergent collective behavior which is not the simple summation of its parts~\cite{ControllingComplexSystem}.

Model-based control of such systems can be intractable to compute and sensitive to uncertainty~\cite{YU2025126253}.
It has been shown that by imposing locality/separability, 
many large optimal control problems decompose into parallel local subproblems whose complexity does not grow with the total system size, 
using techniques such as system-level synthesis~\cite{wang2017separablelocalizedlevelsynthesis}.

\subsection{Model Order Reduction}

The size and complexity of large scale systems can be hard to deal with.
Model Order Reduction (MOR) methods tackle the problem of reducing their complexity.
This technique attempts to generate
reduced models reproducing the input-output behavior 
of large-scale systems accurately.

Traditional model reduction methods fall into two categories: SVD-based and moment matching methods.
SVD-based techniques are related to the eponymous singular value decomposition.
They core idea is the computation of controllability and observability 
Gramians, eliminating hard-to-reach and -observe states~\cite{lowenerIntro}. 
Their advantages include the preservation of stability and a computable error bound. 
While work is being done to make them more efficient using approximations \cite{numSolutionLargeScale, prajapati2022model}, their resource requirements
remain their main drawback~\cite{lowenerIntro}. A popular method in this class of MOR techniques is Balanced truncation (BT)~\cite{balancedtruncationOriginal}.

Moment matching methods consider the problem of matching the coefficients of power
series expansions of the transfer function at selected points, reducing model reduction
to rational interpolation. They tend to be cheaper to compute than SVD-based methods, 
but do not provide as strong of theoretical guarantees, the preservation of properties depending on factors such as the choice of expansion points~\cite{ANTOULAS200419}.
A prominent example are Krylov subspace methods~\cite{freund2003model}.

%This distinction does not necessarily apply to the techniques themselves, though, but rather to their
%implementation. 


\subsection{Control Architectures}

We have strategies such as decentralised and distributed control.

Decentralized: controllers act independently with no information exchange.
Distributed: controllers communicate to some degree.
Delays are to be considered. 


\subsection{Approaches for Control}

The approaches taken for controlling large scale systems can roughly be categorised into three distinct levels: Node control, Edge control and Structural control~\cite{ControllingComplexSystem}.

\subsubsection{Node Control}

Control at this level concerns itself with the immediate determination of dynamics of a subset of agents at the lowest, microscopic view.

\subsubsection{Edge Control}

On this abstraction level, control works by dynamically adjusting the communication protocol between the individual agents.

\subsubsection{Structural Control}

At the highest level, the network topology can be reshaped in order to guide behaviour at the macroscopic level.

Most works are concerned with node control, as this resembles the traditional single, small-system control the closest.
Best results are usually achieved when considering them all though~\cite{ControllingComplexSystem}.


\section{DDC for Linear Systems}


\subsection{Data-driven Model Reduction}

When a linear model is too large to identify and use for control directly, 
one can use data-driven model reduction.
Data-driven MOR methods rely only on measured trajectories, frequency response data, or impulse responses.


\subsubsection{Interpolatory approaches}

A widely-used approach is data-driven interpolatory model reduction.
One such method, the Loewner framework, constructs a reduced linear
model from transfer-function data or sampled impulse responses~\cite{gosea2021datadrivenmodelingcontrollargescale}.
The main element of the Loewner framework is the Loewner Matrix. For two lists of complex number pairs \(u_i, v_i\) and \(\mu_j, \lambda_j\), it is defined as follows ~\cite{lowenerIntro}: 

$$\begin{pmatrix}
\frac{v_1-w_1}{\mu_1-\lambda_1} & \dots & \frac{v_1-w_k}{\mu_1 - \lambda_k}\\
\dots & \dots & \dots \\
\frac{v_q-w_1}{\mu_q-\lambda_1} & \dots & \frac{v_q-w_k}{\mu_q - \lambda_k}
\end{pmatrix}$$


The rank of this Loewner matrix then contains information about the minimal admissible complexity of the solutions of the interpolation.
The shifted Loewner matrix \(L_\sigma\) has the entries \(L_{\sigma_{ij}} = \frac{\mu_jv_j- \lambda_iw_i}{\mu_j-\lambda_i}\).
The key insight is that for a linear system, 
these matrices are directly related to the system's internal  % double check this - chatgpt
state-space matrices without needing to know them explicitly~\cite{karachalios2023data}.
The singular values of the Loewner matrix are used to determine the numerical 
rank of the system, which suggests the appropriate order of the reduced model.  
A sharp drop-off in the singular values indicates a good candidate for the 
reduced model's dimension. 
By selecting a small number of the largest singular values, 
one can capture the most important dynamics of the original system.
The reduced-order model's state-space matrices
are then directly computed from the projected Loewner matrices 
and the data vectors~\cite{lowenerIntro}. % double chekck


Another interpolatory MOR technique is the AAA algorithm~\cite{nakatsukasa2018aaa}, which builds rational
approximants adaptively by selecting 
interpolation points based on approximation error. 
This method is attractive when only frequency response data is available, 
as it balances accuracy and model complexity with minimal user intervention.
% How does it work??

\subsubsection{Balanced truncation methods}


Balanced truncation (BT)~\cite{balancedtruncationOriginal} is a widely successfully employed MOR technique for large systems~\cite{salehi2021mixed}, traditionally relying on controllability and observability Gramians and hence access to system matrices.
A data-driven reformulation is presented in \cite{gosea2021datadrivenbalancinglineardynamical}.
The fundamental insight driving this development is that BT does not use the Gramians directly,
but rather their product, which can be approximated from data. The results of the numerical BT method 
can get arbitrary close to those of classical BT, depending on how much computational resources are allocated to the numerical quadrature.



% some more, see below
\subsubsection{Optimality-based approaches}

For discrete-time systems, \cite{sakamoto2024data} proposes 
a data-driven \(h^2\) model reduction technique leveraging the
discrete-time Lyapunov and Sylvester equations, deriving 
the data-driven gradient.

\subsubsection{Modal decomposition methods}

Proper orthogonal decomposition (POD) is one of the most widely used dimensionality-reduction techniques in dynamical systems.
The key idea is to approximate system trajectories by 
projecting them onto a low-dimensional basis obtained directly from data.
For a dataset \(X = [x_1x_2...x_m], x_i \in \mathbb{R}^n\), POD seeks an orthonormal basis
\(U_r\in\mathbb{R}^{n\times r}\) that minimises the projection error 

\begin{equation}
\min_{U_r^TU_r = I_r} ||X-U_rU_r^TX||^2_F
\end{equation}

This optimization problem is solved by computing the singular value decomposition of the snapshot matrix \(X\). 
The dominant singular vectors define the reduced basis, and the singular values quantify the energy captured by each mode~\cite{eskew2023new}.


A closely related algorithm is Dynamic Mode Decomposition is Dynamic Mode Decomposition (DMD).
Given snapshot pairs \(S^- = [x_1, ... , x_{m-1}]\) and \(S^+ = [x_2, ..., x_{m}]\), DMD approximates the system by finding
a matrix \(A\) such that 

\begin{equation}
x_{k+1} \approx Ax_k
\end{equation}

The goal is to determine a low-rank subspace of \(A\) and through it represent system dynamics.
One first performs SVD on \(S^- = U\Sigma V^T\), truncating \(U\) and \(V\) to the first \(r\) columns and selecting the first \(r\) singular values from \(\Sigma\).
What value to use for \(r\) depends on accuracy requirements, though a typical choice is to have it retain
a specific fraction of information, such that

\begin{equation}
r = \argmin_j \frac{\sum_{i=1}^{j}\sigma_i}{\sum_{i=1}^n\sigma_i} < \tau
\end{equation}

One can then reconstruct \(A_r = U_rS^+V_r\Sigma_r^{-1}\).
Using the cheap eigen-decomposition \(A_rW = \Lambda W\), one can compute the DMD modes \(\Psi = U_rW\), which 
contain the essential information about the system's evolution and can be used to approximately predict the system's state at any point in time~\cite{HUHN2023111852}.

%Other data-driven methods, such as
%proper orthogonal decomposition (POD) [26], dynamic mode
%decomposition (DMD) [27], [28], or operator inference [29],
%[30] can also be employed. 

%The datadriven balanced truncation method [13]–[16] is one of the
%common data-driven model reduction inspired by the balanced
%truncation method, which is a model-based MOR. In this
%method, balanced truncation is performed by using the measurement data. For example, [15] performs MOR by estimating
%the Gramians from the data information. 

%Another well-known
%approach is data-driven interpolatory model reduction. This
%approach includes the Loewner framework [17]–[23] and the
%AAA algorithm [24], [25]. Other data-driven methods, such as
%proper orthogonal decomposition (POD) [26], dynamic mode
%decomposition (DMD) [27], [28], or operator inference [29],
%[30] can also be employed. 
%https://arxiv.org/pdf/2401.05774

\subsection{Decentralised and Distributed Control}


\section{DDC for Nonlinear and Complex Systems}


\subsection{Koopman based and Lifting Methods}

One approach is to lift the nonlinear system into 
a higher-dimensional (possibly infinite) linear space via the Koopman operator, then apply linear data methods.
Recent research has focused on developing a Koopman MPC technique where one learns a linear 
predictor (via Extended Dynamic Mode Decomposition or neural nets) and then solves a convex MPC on the lifted space~\cite{LI2025100219}.


\section{Data-driven Higher Level Control}


While most data-driven control efforts for 
large-scale systems focus on the microscopic 
level of node control, there is increasing 
recognition that higher levels of abstraction 
can provide leverage for scalability and robustness. 
Specifically, edge control and structural control 
have emerged as two promising approaches to orchestrating 
collective behavior in complex systems~\cite{ControllingComplexSystem}. 
These approaches shift attention from 
direct state manipulation of individual nodes to 
the design of interactions and the adaptation 
of system topology, providing a macroscopic handle 
on global dynamics.


\subsection{Data-driven Edge Control}

\subsection{Data-driven Structural Control}



%\begin{figure}[h]
%\begin{center}
%\includegraphics{./images/grins.pdf}
%\end{center}
%\caption{A vector graphic loaded from a PDF file}
%\label{Pic1}
%\end{figure}

%\begin{figure}[h]
%\begin{center}
%\includegraphics{./images/grins.png}
%\end{center}
%\caption{A bitmap graphic loaded from a PNG file}
%\label{Pic2}
%\end{figure}


% An example of a floating figure using the graphicx package.
% Note that \label must occur AFTER (or within) \caption.
% For figures, \caption should occur after the \includegraphics.
% Note that IEEEtran v1.7 and later has special internal code that
% is designed to preserve the operation of \label within \caption
% even when the captionsoff option is in effect. However, because
% of issues like this, it may be the safest practice to put all your
% \label just after \caption rather than within \caption{}.
%
% Reminder: the "draftcls" or "draftclsnofoot", not "draft", class
% option should be used if it is desired that the figures are to be
% displayed while in draft mode.
%
%\begin{figure}[!t]
%\centering
%\includegraphics[width=2.5in]{myfigure}
% where an .eps filename suffix will be assumed under latex, 
% and a .pdf suffix will be assumed for pdflatex; or what has been declared
% via \DeclareGraphicsExtensions.
%\caption{Simulation Results.}
%\label{fig_sim}
%\end{figure}

% Note that IEEE typically puts floats only at the top, even when this
% results in a large percentage of a column being occupied by floats.


% An example of a double column floating figure using two subfigures.
% (The subfig.sty package must be loaded for this to work.)
% The subfigure \label commands are set within each subfloat command,
% and the \label for the overall figure must come after \caption.
% \hfil is used as a separator to get equal spacing.
% Watch out that the combined width of all the subfigures on a 
% line do not exceed the text width or a line break will occur.
%
%\begin{figure*}[!t]
%\centering
%\subfloat[Case I]{\includegraphics[width=2.5in]{box}%
%\label{fig_first_case}}
%\hfil
%\subfloat[Case II]{\includegraphics[width=2.5in]{box}%
%\label{fig_second_case}}
%\caption{Simulation results.}
%\label{fig_sim}
%\end{figure*}
%
% Note that often IEEE papers with subfigures do not employ subfigure
% captions (using the optional argument to \subfloat[]), but instead will
% reference/describe all of them (a), (b), etc., within the main caption.


% An example of a floating table. Note that, for IEEE style tables, the 
% \caption command should come BEFORE the table. Table text will default to
% \footnotesize as IEEE normally uses this smaller font for tables.
% The \label must come after \caption as always.
%
%\begin{table}[!t]
%% increase table row spacing, adjust to taste
%\renewcommand{\arraystretch}{1.3}
% if using array.sty, it might be a good idea to tweak the value of
% \extrarowheight as needed to properly center the text within the cells
%\caption{An Example of a Table}
%\label{table_example}
%\centering
%% Some packages, such as MDW tools, offer better commands for making tables
%% than the plain LaTeX2e tabular which is used here.
%\begin{tabular}{|c||c|}
%\hline
%One & Two\\
%\hline
%Three & Four\\
%\hline
%\end{tabular}
%\end{table}


% Note that IEEE does not put floats in the very first column - or typically
% anywhere on the first page for that matter. Also, in-text middle ("here")
% positioning is not used. Most IEEE journals/conferences use top floats
% exclusively. Note that, LaTeX2e, unlike IEEE journals/conferences, places
% footnotes above bottom floats. This can be corrected via the \fnbelowfloat
% command of the stfloats package.



\section{Conclusion}


% conference papers do not normally have an appendix


% use section* for acknowledgement
% \section*{Acknowledgment}
% The authors would like to thank...





% trigger a \newpage just before the given reference
% number - used to balance the columns on the last page
% adjust value as needed - may need to be readjusted if
% the document is modified later
%\IEEEtriggeratref{8}
% The "triggered" command can be changed if desired:
%\IEEEtriggercmd{\enlargethispage{-5in}}



% references section

% can use a bibliography generated by BibTeX as a .bbl file
% BibTeX documentation can be easily obtained at:
% http://www.ctan.org/tex-archive/biblio/bibtex/contrib/doc/
% The IEEEtran BibTeX style support page is at:
% http://www.michaelshell.org/tex/ieeetran/bibtex/
%\bibliographystyle{IEEEtran}
% argument is your BibTeX string definitions and bibliography database(s)
%\bibliography{IEEEabrv,../bib/paper}
%
% <OR> manually copy in the resultant .bbl file
% set second argument of \begin to the number of references
% (used to reserve space for the reference number labels box)

\printbibliography

%\begin{thebibliography}{1}

%\bibitem{IEEEhowto:kopka}
%H.~Kopka and P.~W. Daly, \emph{A Guide to \LaTeX}, 3rd~ed.\hskip 1em plus
%  0.5em minus 0.4em\relax Harlow, England: Addison-Wesley, 1999.

%\end{thebibliography}




% that's all folks
\end{document}


